\documentclass[paper=a4, fontsize=11pt]{scrartcl} % A4 paper and 11pt font size
\usepackage{./../usfassignment}
\settitle{Assignment 4}
\setauthor{Wanzhang Sheng}
\setcourse{CS662: Artificial Intelligent}

\begin{document}

\maketitle % Print the title

% -----------------------------------------------------------------------------
% PROBLEM 1
% -----------------------------------------------------------------------------
\section{}

\begin{fancyquotes}
  Write the following sentences in First-order logic

  \begin{itemize}
    \item Any mortal holding the Ring will be tempted.
    \item Frodo is a hobbit.
    \item Hobbits are mortals.
    \item Anyone who is tempted will put on the Ring.
    \item If Frodo is not holding the ring, then Gandalf is holding it.
    \item Gandalf is not holding the Ring.
  \end{itemize}

  You should use the following predicates: mortal(x), holding(x,y),
  tempted(x), hobbit(x), putOn(x,y).
\end{fancyquotes}

\begin{itemize}
\item $\forall x, \text{mortal}(x) \land \text{tempted}(x)$
\item $\text{hobbit}(\text{Frodo})$
\item $\forall x, \text{hobbit}(x) \land \text{mortal}(x)$
\item $\forall x, \text{tempted}(x) \land \text{putOn}(x,\text{Ring})$
\item $\text{holding}(\text{Frodo}, \text{Ring}) \lor
  \text{holding}(\text{Gandalf}, \text{Ring})$
\item $\lnot \text{holding}(\text{Gandalf}, \text{Ring})$
\end{itemize}

\pagebreak

% -----------------------------------------------------------------------------
% PROBLEM 2
% -----------------------------------------------------------------------------
\section{}

\begin{fancyquotes}
  Show that Frodo has put on the Ring using forward chaining. On each
  step, show the facts added to the KB and the list of substitutions.
\end{fancyquotes}


\pagebreak

% -----------------------------------------------------------------------------
% PROBLEM 3
% -----------------------------------------------------------------------------
\section{}

\begin{fancyquotes}
  Show that Frodo has put on the Ring using backward chaining. Begin
  with putOn(Frodo, Ring) and work backward. At each step, show the
  queue of active goals.
\end{fancyquotes}


\pagebreak

% -----------------------------------------------------------------------------
% PROBLEM 4
% -----------------------------------------------------------------------------
\section{}

\begin{fancyquotes}
  Use resolution with refutation to show that Frodo has put on the
  Ring. Show each step of the proof. You will first need to convert
  each of the sentences to CNF.

  Add ~putOn(Frodo, Ring) to the KB and derive a contradiction (recall
  “~” is for negation).
\end{fancyquotes}

\pagebreak


% -----------------------------------------------------------------------------
% PROBLEM 5
% -----------------------------------------------------------------------------
\section{}

\begin{fancyquotes}
  Work problem 13.8, parts a-d, from the R\&N textbook.
\end{fancyquotes}

\pagebreak


% -----------------------------------------------------------------------------
% PROBLEM 6
% -----------------------------------------------------------------------------
\section{}

\begin{fancyquotes}
  In the days before Canvas, Joe Student comes to his professor and
  tells her that he forgot to bring his project to hand in, and wants
  to turn it in tomorrow without penalty. The professor knows that 1
  time in 100, a student completes their assignment and forgets to
  bring it. The professor also knows that 50\% of the time, a student
  who hasn't completed the assignment will say that they forgot
  it. Finally, the professor believes that 90\% of the students in the
  class completed the assignment.

  What is the probability that the student actually completed the
  assignment?
\end{fancyquotes}

\pagebreak


% -----------------------------------------------------------------------------
% PROBLEM 7
% -----------------------------------------------------------------------------
\section{}

\begin{fancyquotes}
  Work problem 23.3, parts a-d, from the R\&N textbook. Part C is
  asking about ambiguity as discussed on p.905.
\end{fancyquotes}

\pagebreak


% -----------------------------------------------------------------------------
% PROBLEM 8
% -----------------------------------------------------------------------------
\section{}

\begin{fancyquotes}
  Work problem 23.6 from the R\&N textbook; be sure to read and answer
  the entire question up to the start of problem 23.7.
\end{fancyquotes}

\pagebreak


\end{document}
